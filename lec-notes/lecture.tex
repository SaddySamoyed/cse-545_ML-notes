\documentclass[lang=cn,11pt]{elegantbook}
\usepackage[utf8]{inputenc}
\usepackage[UTF8]{ctex}
\usepackage{amsmath}%
\usepackage{amssymb}%
\usepackage{graphicx}
\usepackage{pdfpages}

\title{CSE 545: Machine Learning}
\subtitle{25 Win, instructed by Honglak Lee}

\begin{document}
\frontmatter
\tableofcontents
\mainmatter

\chapter{Linear Regression: I}

\begin{definition}{supervised learning}
\textbf{Supervise Learning}: Given data X in feature space and labels Y, learn to predict Y from given X.    
\end{definition}
\noindent Label: 可以是 discrete 的 or continuous 的.\\
对于\textbf{discrete} 的 label, 这类问题称为 \textbf{classification}.\\
对于\textbf{continuous} 的 label, 这类问题称为 \textbf{regression}.\\


\section{notation and expression}
我们使用以下 notation:
\pic[0.7]{assets/lec1-notation.png}

\begin{definition}{(generalized) linear regression}
给定 $N$ 个 data points $\{(x^{(n)},y^{(n)}) \}_{n=1,\cdots, N}$ where each $x^{(n)}\in\bR^d,y^{(n)}  \in\bR$, \\
以及预先设定好的 $M$ 个 basis functions $\{ \phi_i(x)\}_{i=1,\cdots, M}$ 用以表示 $M$ 个 features,\\\
   
我们通过建立一个 $h(x,w): \bR^d \times \bR^M \rar \bR = \sum_{i=0}^{M-1} w_i \phi_i(x)$, 使其关于 $w$ 线性, 以找到一组参数 $w \in \bR^M$, 使得 $h(x^{(n)},w)$ 能够近似 $y^{(n)}$ for each $n$, with respect to the loss function we define to measure the distance between two vectors. 
\end{definition}
\pic[0.7]{assets/lec1(expression).png}
\begin{remark}
    注意: linear regression 指的是 $y\in\bR$ 和参数 $w \in \bR^M$ 之间是 linear 的, 而不是说 $y$ 和 input $x$ 之间是 linear 的. 我们可以选择 nonlinear 的 basis funtions 来 encode $x$ 来表示 features 的特性, 比如我们可以选择:
\pic[0.7] {assets/lec1-basis.png}
\end{remark}




\section{loss function: sum of squared error}
这个 loss function 衡量两个 vectors 之间的距离, 目的是衡量 $y \in \bR^N$ 和 $h(x,w) \in \bR^N$ 这两个 vectors 的差距. 实际上就是它们 difference 的 $L_2$-norm 的平方.
\pic[0.6]{assets/lec1-loss.png}




\section{gradient of sum of squared error}
我们下面首先通过求 $\nabla E(w)$ 的每个 entry $\frac{\partial E}{\partial w_k}(w)$ 来写出这个 gradient.
\pic[0.6]{assets/lec1-gradient.png}




\section{batch v.s. stochastic GD}

我们通过迭代降低 gradient 来降低 loss function 的值, 从而优化 weight vector.
\pic[0.8]{assets/lec1-descent.png}

More practically, 我们可以采用 minibatch SGD: 即在 batch GD 和 SGD 之间, 每次选择一小部分 samples, 称为一个 \textbf{minibatch}, 在这个 minibatch 上进行 GD.







\chapter{Linear Regression: II}
\section{vectorization}
我们可以把每个 $x^{(n)}$ 的 features 写成一个 row vector, 并 stack up $N$ 个 row vectors, 成为一个 $N\times M$ 的 matrix $\Phi$. 从而:
$$
h(x,w) = \Phi w
$$
vectorization 的好处是: 1. 便于手算; 2. computer 可以进行并行计算.
\pic[0.6]{assets/lec2-vect.png}
计算得 linear regression 的 loss function 为:
\[
E(w) = \frac{1}{2}w^T \Phi^T \Phi w - w^T \Phi^T y + \frac{1}{2}y^T y
\]



\section{closed-form solution}
如果
$$
\nabla E(w) = y
$$
有一个 closed form solution, 那么这个 solution 一定是一个 local min/max, 从而 possibly 成为一个 global min.\\
为了计算 closed form solution, 我们首先要给出 $\nabla E(w)$ 的 matrix form 表达式. \\
这里首先引入 linear form 和 quadratic form ($\bR^m \rar \bR$) 的 gradient:
\pic[0.6]{assets/lec2-diff.png}
我们发现: $E(w)$ 就是一个 $w$ 的 quadratic form, 一个 $w$ 的 linear form 和一个 const 的组合. 从而可以求出:
\pic[0.6]{assets/lec2-grad.png}
从而我们得到 closed form solution (if exists):
\pic[0.6]{assets/lec2-closed.png}
因而 closed form exists iff $\Phi^T\Phi$ 可逆, iff $\Phi$ 可逆.\\
并且 recalll in linear algebra: $\rank(\Phi^T\Phi) = \rank(\Phi)$.
因而, \textbf{closed form exists iff $M >= N $ 且 $\rank(\Phi) = N$.}






\hypertarget{overfitting}{%
\section{overfitting}\label{overfitting}}

\pic[0.6]{ Linear_Regression.assets/Screenshot 2025-01-21 at 18.11.31.png}

\pic[0.6]{ Linear_Regression.assets/Screenshot 2025-01-21 at 18.11.37.png}

overfitting 的原因: features 数量 M 设置得太多, 导致过度保持 training
sets 的点靠近曲线, 但是对于 testing set 并不对( 这里是一个简化,
实则不能单纯这样划分, 需要 cross validation)

\pic[0.6]{ Linear_Regression.assets/Screenshot 2025-01-21 at 18.11.50.png}

overfitting 的表现: 各项 features 的参数动荡很大.

\pic[0.6]{ Linear_Regression.assets/Screenshot 2025-01-21 at 18.12.41.png}

overfitting 的解决方法 1: 增加数据点

\pic[0.6]{ Linear_Regression.assets/Screenshot 2025-01-21 at 18.13.03.png}

overfitting 的解决方法 2:

\hypertarget{regularization-solving-overfit}{%
\subsection{regularization: solving
overfit}\label{regularization-solving-overfit}}

我们通过引入一个 regularization term, 也称为 penalty term 惩罚项,
以使得曲线尽量平缓, 从而减少 overfitting.

Idea: 把 \(w\) 本身的 Magnitute 作为一个 loss function 的一部分,
让我们降低 loss 的同时自带降低 w 的各个 entries 的正负动荡程度,
从而使得拟合曲线尽量平缓, 降低曲线的 expressibility.

\pic[0.6]{ Linear_Regression.assets/image-20250121201523836.png}

这里的 \(\lambda\) 理应设置较小, 如 0.001 等.

\(\lambda\) 设置越大, 曲线越接近 constant. 比如 \(\lambda := 1\), 则会

\pic[0.6]{ Linear_Regression.assets/Screenshot 2025-01-21 at 20.17.21.png}

如果 traning error 和 testing error 都很大, 那就说明 \(\lambda\)
调太大了.

\hypertarget{gradient-of-regularized-least-square}{%
\subsection{gradient of regularized least
square}\label{gradient-of-regularized-least-square}}

\pic[0.6]{ Linear_Regression.assets/image-20250121204803377.png}

summary: regularization controls the tradeoff bewteen fitting error 和
expressibility.

\hypertarget{linear-regressionlec3}{%
\chapter{Linear Regression(lec3)}\label{linear-regressionlec3}}

\hypertarget{review-on-probability}{%
\section{Review on Probability}\label{review-on-probability}}

\pic[0.6]{ Linear_Regression.assets/image-20250121215444693.png}

\hypertarget{likelihood-function}{%
\subsection{Likelihood function}\label{likelihood-function}}

\textbf{Likelihood function} (似然函数) 表示在给定一组 i.i.d 的数据
samples \(D\) 以及其以 \(\theta\) 为参数的分布形式下,random vector
=\(D\) 处的概率密度;其以参数 \(\theta\) 为变量,表达的是在固定数据
\(D\) 的前提下,不同参数 \(\theta\) 对数据的适配程度

而 maximum likelihood estimator 则是使得这个 random vector =\(D\)
处的概率密度 maximize 的参数 \(\tilde{\theta}\)

由于独立同分布, random vector =\(D:=\{x_1,\cdots,x_n\}\)
处的概率密度就等于所有 \(X=x_i\) 处的概率密度的 product.

取得 maximum likelihood estimator 即: 在这个参数下, 我们得到的模型,
对于我们的训练数据而言, 取得其相对的 y 的概率密度最大.

\pic[0.6]{ Linear_Regression.assets/image-20250123174328307.png}

\hypertarget{find-mle-for-linear-model-with-stochastic-noise}{%
\section{Find MLE for linear model with stochastic
noise}\label{find-mle-for-linear-model-with-stochastic-noise}}

\pic[0.6]{ Linear_Regression.assets/image-20250123180811150.png}

(求解MLE 可得: linear model with stochastic noise which is normal
distributed centered at 0 得到的 MLE,与标准的 linear model 得到的 MLS
最优解是等价的.

这是符合直觉的,因为一个正态分布的 noise 不影响参数的选择)

\pic[0.6]{ Linear_Regression.assets/image-20250123180955714.png}

\hypertarget{locally-weighted-linear-regression}{%
\section{\texorpdfstring{locally weighted linear regression
}{locally weighted linear regression }}\label{locally-weighted-linear-regression}}

\pic[0.6]{ Linear_Regression.assets/image-20250123193833909.png}

















\end{document}